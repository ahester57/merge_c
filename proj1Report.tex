
% CS 4710 HW 1

\documentclass{article}
\usepackage{titling}
\usepackage{siunitx}	% for scientific notation

\setlength{\droptitle}{-15em}

\begin{document}

\title{CS3130 - Project 1}
\author{Austin Hester}
\date{Feb. 22, 2017}
\maketitle


% This makes it so sections aren't automatically numbered
\makeatletter
\def\@seccntformat#1{%
	 \expandafter\ifx\csname c@#1\endcsname\c@section\else
	  \csname the#1\endcsname\quad
  \fi}
\makeatother

%%%	Begin		%%%

%% Results Section %%

\section{Results}

\begin{center}
\begin{tabular}{ | l | c | r | }
	\hline
	 Elements & Merge-sort & Bubble-sort \\ \hline  \hline
	1500 & 0.000s & 0.000s \\
	10k & 0.000s & 0.063s \\
	25k & 0.015s & 0.406s \\
	50k & 0.031s & 1.578s \\
	75k & 0.031s & 3.610s \\
	150k & 0.062s & 14.344s \\
	250k & 0.109s & 39.969s \\
	500k & 0.218s & 159.858s \\
	1m & 0.438s & 639.656s \\
	2m & 0.922s & 2571.156s \\
	\hline
\end{tabular}
\end{center}


%% Analysis Section %%

\section{Analysis}

%	Bubble	%

\subsection{Bubble}

Bubble sort took (literally) exponentially more time as the number of elements increased. \\

	Based on $T(n) = \theta (n^2)$, I got a "sort constant," $c \approx \num{1.566e9}$ from $\frac{n^2}{t}$ \\

Although this "sort constant" varies slightly from run to run, it can be used to successfully predict the time a certain
execution will take. For example, I was able to predict a bubble sort of 2 million integers would take 2547s using
the equation $t \approx \frac{n^2}{1.56e9}$.

\subsubsection{Worst-case}

Bubble-sort's worst case is when its elements are in reverse order. \\

	$T(n) = \theta (n^2)$

\subsubsection{Best-case}

Bubble-sort shines when it comes to one thing, sorting already sorted lists. \\
In this case the algorithm goes through the list once, and does not make a switch. It then 
terminates based on the sentinel, when no replacements occur. \\

	$T(n) = \theta (n)$

\subsubsection{Average-case}

Average case 

\newpage

%	Merge		%

\subsection{Merge}

Merge sort never exceeded 1 second, even when sorting 2 million integers.\\

From $T(n) = \theta (n \lg{n})$,the "sort constant," $c \approx \num{4.270e7}$ from $\frac{n \lg{n}}{t}$

\subsubsection{Worst-case}

Worst case is: \\

	$T(n) = \theta (n \lg{n})$

\subsubsection{Best-case}

Best case is: \\

	$T(n) = \theta (n \lg{n})$

\subsubsection{Average-case}

Average case is: \\

	$T(n) = \theta (n \lg{n})$

%% Conclusion %%

\section{Conclusion}

Don't use bubble sort unless you are using it to check to make sure an seemingly sorted list is completely sorted.

\end{document}














